
---

# **📖 Ratio & Proportion Comprehensive Revision Notes** 🎯  

---

### **1️⃣ Volume Formulas**  
In certain ratio-based problems, understanding volume formulas is essential. Here are two common ones:  
- **Cylinder:**  
\[
\text{Volume} = \pi r^2 h
\]  
- **Cone:**  
\[
\text{Volume} = \frac{1}{3} \pi r^2 h
\]  
Knowing these formulas aids in solving ratio-based geometry problems efficiently.

---

### **2️⃣ Property of Equal Ratios**  
If two ratios are equal:  
\[
\frac{a}{b} = \frac{c}{d}
\]  
Then, cross-multiplication gives:  
\[
a \times d = b \times c
\]  
This property is fundamental for solving proportion-related problems.

---

### **3️⃣ Modified Property with Coefficients**  
When ratios include coefficients, simplify first by removing the coefficients and applying cross-multiplication rules. This ensures clarity and reduces complexity.

---

### **4️⃣ Duplicate Ratio**  
Duplicate ratio refers to the ratio of squares of two numbers.  
\[
\text{If } a : b, \text{then duplicate ratio is } a^2 : b^2.
\]

---

### **5️⃣ Sub-Duplicate Ratio**  
The sub-duplicate ratio represents the ratio of the square roots of two numbers.  
\[
\text{If } a : b, \text{then sub-duplicate ratio is } \sqrt{a} : \sqrt{b}.
\]

---

### **6️⃣ Triplicate Ratio**  
The triplicate ratio is the ratio of cubes of two numbers.  
\[
\text{If } a : b, \text{then triplicate ratio is } a^3 : b^3.
\]

---

### **7️⃣ Sub-Triplicate Ratio**  
This ratio involves cube roots.  
\[
\text{If } a : b, \text{then sub-triplicate ratio is } \sqrt[3]{a} : \sqrt[3]{b}.
\]

---

### **8️⃣ Inverse/Reverse Ratio**  
The inverse ratio flips the original terms.  
\[
\text{If } a : b, \text{then inverse ratio is } b : a.
\]

---

### **9️⃣ Compound Ratio**  
A compound ratio combines two ratios.  
\[
\text{For } a : b \text{ and } c : d, \text{compound ratio is } (a \times c) : (b \times d).
\]

---

### **🔟 Finding Ratios When One Value is More Than Another**  
These problems often require you to calculate the difference or surplus between values. Use basic subtraction or addition and then find the ratio.

---

### **11️⃣ Padosa Technique**  
#### **Padosa Horizon** *(Content to be added by Tanya)*  

The Padosa Technique will be explained in detail later, along with its sub-topic.

---

### **12️⃣ Componendo & Dividendo Rule**  
📌 **Formula:**  
If  
\[
\frac{a}{b} = \frac{c}{d}
\]  
Then:  
\[
\frac{a + b}{a - b} = \frac{c + d}{c - d}
\]  
✅ **Key Insight:** Useful for simplifying ratio-based problems, especially under time constraints.

---

### **13️⃣ Convertendo Property**  
📌 **Formula:**  
If  
\[
\frac{a}{b} = \frac{c}{d}
\]  
Then:  
\[
\frac{a}{a - b} = \frac{c}{c - d}
\]  
✅ **Key Insight:** Helps in transforming proportions for faster problem-solving.

---

### **14️⃣ Product of Extremes Property**  
📌 **Formula:**  
If  
\[
a : b :: c : d
\]  
Then:  
\[
a \times d = b \times c
\]  
✅ **Key Insight:** Enables quick solution of proportional equations.

---

### **15️⃣ Continuous Proportion**  
📌 **If \( a, b, c \) are in continuous proportion**, then:  
✅ **Mean Proportion**:  
\[
b = \sqrt{a \times c}
\]  
✅ **Third Proportion**:  
\[
c = \frac{b^2}{a}
\]  
✅ **Key Insight:** Helps identify relationships in sequential ratios.

---

### **16️⃣ Income & Expenditure Ratio Question**  
📌 **Problem Statement:**  
The ratio of incomes of A & B is **3:5**, and the ratio of expenditures is **1:2**. If each saves **₹4000**, find the incomes of A & B.  

✅ **Shortcut Trick:**  
- Cross multiplication results denominator  
- Expenditure difference is numerator  
- Numerator × Savings  
- Multiply by income ratio values to find A & B’s incomes.

---

### **17️⃣ Comparing Ratios**  
📌 **Method:**  
1. Find the LCM of denominators.  
2. Adjust both fractions to have the same denominator.  
3. Compare numerators to determine which ratio is greater or smaller.

---

### **18️⃣ PDSD Method** *(Product Difference & Sum Difference)*  
📌 **Question:**  
What should be added to each of **5, 9, 11, 18** to make them proportional?  

✅ **Shortcut Trick:**  
1. Pair smallest & largest numbers.  
2. Pair remaining two numbers.  
3. Numerator = Difference between product of pairs.  
4. Denominator = Difference between sum of pairs.  
5. Final answer = Value to be added.

---

### **19️⃣ Ratio of Sum & Difference of Two Numbers**  
📌 **Question:**  
The ratio of the **sum and difference** of two numbers is **9:2**. Find the ratio of the numbers using **Componendo & Dividendo**.  

✅ **Shortcut Solution:**  
\[
\frac{a}{b} = \frac{11}{7}
\]  

---

### **20️⃣ Some Very Basic Common Sense Questions**  
📌 **1️⃣ What should be subtracted from each term of 15:19 so that the new ratio becomes 3:4?**  
📌 **2️⃣ What should be added to each term of 9:16 so that the new ratio becomes 2:3?**  

---

### **21️⃣ Income Formula**  
✅ **Income = Savings + Expenditure**  

📌 **Question:**  
The ratio of income of **Priyanka Chopra** and **Hrithik Roshan** is **3:4**. The ratio of their expenditures is **4:5**. Find the ratio of their savings if Priyanka’s savings are **one-fourth of her income**.  

---

### **22️⃣ Key Note on Solving Ratio & Proportion Problems**  
📌 Solving ratio-based problems (especially salary-related ones) involves forming equations step by step based on what we need.

---

✅😊📚
