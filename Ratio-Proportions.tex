
---

# 📖 Ratio & Proportion Comprehensive Revision Notes 🎯  

## 1️⃣ Volume Formulas  
In certain ratio-based problems, understanding volume formulas is essential. Here are two common ones:  

- **Cylinder:**  
  $$ \text{Volume} = \pi r^2 h $$  

- **Cone:**  
  $$ \text{Volume} = \frac{1}{3} \pi r^2 h $$  

Knowing these formulas aids in solving ratio-based geometry problems efficiently.  

---

## 2️⃣ Property of Equal Ratios  
If two ratios are equal:  
$$ \frac{a}{b} = \frac{c}{d} $$  
Then, cross-multiplication gives:  
$$ a \times d = b \times c $$  

This property is fundamental for solving proportion-related problems.  

---

## 3️⃣ Modified Property with Coefficients  
When ratios include coefficients, simplify first by removing them and applying cross-multiplication rules.  

---

## 4️⃣ Duplicate Ratio  
Duplicate ratio refers to the ratio of squares of two numbers:  
$$ \text{If } a : b, \text{then duplicate ratio is } a^2 : b^2 $$  

---

## 5️⃣ Sub-Duplicate Ratio  
The sub-duplicate ratio represents the ratio of the square roots of two numbers:  
$$ \text{If } a : b, \text{then sub-duplicate ratio is } \sqrt{a} : \sqrt{b} $$  

---

## 6️⃣ Triplicate Ratio  
The triplicate ratio is the ratio of cubes of two numbers:  
$$ \text{If } a : b, \text{then triplicate ratio is } a^3 : b^3 $$  

---

## 7️⃣ Sub-Triplicate Ratio  
This ratio involves cube roots:  
$$ \text{If } a : b, \text{then sub-triplicate ratio is } \sqrt[3]{a} : \sqrt[3]{b} $$  

---

## 8️⃣ Inverse/Reverse Ratio  
The inverse ratio flips the original terms:  
$$ \text{If } a : b, \text{then inverse ratio is } b : a $$  

---

## 9️⃣ Compound Ratio  
A compound ratio combines two ratios:  
$$ \text{For } a : b \text{ and } c : d, \text{compound ratio is } (a \times c) : (b \times d) $$  

---

## 🔟 Finding Ratios When One Value is More Than Another  
Use subtraction or addition based on differences between values to find the ratio.  

---

## 1️⃣1️⃣ Padosa Technique  
Padosa Horizon *(Content to be added by Tanya)*  

---

## 1️⃣2️⃣ Componendo & Dividendo Rule  
📌 **Formula:**  
If  
$$ \frac{a}{b} = \frac{c}{d} $$  
Then:  
$$ \frac{a + b}{a - b} = \frac{c + d}{c - d} $$  

✅ **Key Insight:** Useful for simplifying ratio-based problems!  

---

## 1️⃣3️⃣ Convertendo Property  
📌 **Formula:**  
If  
$$ \frac{a}{b} = \frac{c}{d} $$  
Then:  
$$ \frac{a}{a - b} = \frac{c}{c - d} $$  

✅ **Key Insight:** Helps in transforming proportions for faster problem-solving.  

---

## 1️⃣4️⃣ Product of Extremes Property  
📌 **Formula:**  
If  
$$ a : b :: c : d $$  
Then:  
$$ a \times d = b \times c $$  

✅ **Key Insight:** Enables quick solutions in proportions.  

---

## 1️⃣5️⃣ Continuous Proportion  
📌 **If** \( a, b, c \) **are in continuous proportion**, then:  
✅ **Mean Proportion:**  
$$ b = \sqrt{a \times c} $$  
✅ **Third Proportion:**  
$$ c = \frac{b^2}{a} $$  

✅ **Key Insight:** Helps in sequential ratio relationships.  

---

## 1️⃣6️⃣ Income & Expenditure Ratio Question  
📌 **Problem Statement:**  
The ratio of incomes of A & B is **3:5**, and the ratio of expenditures is **1:2**. If each saves **₹4000**, find the incomes of A & B.  

✅ **Shortcut Trick:**  
- Cross multiplication results denominator  
- Expenditure difference is numerator  
- Numerator × Savings  
- Multiply by income ratio values  

---

## 1️⃣7️⃣ Comparing Ratios  
📌 **Method:**  
- Find the LCM of denominators.  
- Adjust both fractions.  
- Compare numerators.  

---

## 1️⃣8️⃣ PDSD Method *(Product Difference & Sum Difference)*  
📌 **Question:**  
What should be added to each of **5, 9, 11, 18** to make them proportional?  

✅ **Shortcut Trick:**  
1. Pair smallest & largest numbers.  
2. Pair remaining two numbers.  
3. Find numerator = Difference between product of pairs.  
4. Find denominator = Difference between sum of pairs.  

---

## 1️⃣9️⃣ Ratio of Sum & Difference of Two Numbers  
📌 **Question:**  
The ratio of the **sum and difference** of two numbers is **9:2**. Find the ratio of numbers using **Componendo & Dividendo**.  

✅ **Shortcut Solution:**  
$$ \frac{a}{b} = \frac{11}{7} $$  

---

## 2️⃣0️⃣ Some Very Basic Common Sense Questions  
📌 **1️⃣ What should be subtracted from each term of 15:19 so that the new ratio becomes 3:4?**  
📌 **2️⃣ What should be added to each term of 9:16 so that the new ratio becomes 2:3?**  

---

## 2️⃣1️⃣ Income Formula  
✅ **Income = Savings + Expenditure**  

📌 **Question:**  
The ratio of income of **Priyanka Chopra** and **Hrithik Roshan** is **3:4**. The ratio of their expenditures is **4:5**. Find the ratio of their savings if Priyanka’s savings are **one-fourth of her income**.  

---

## 2️⃣2️⃣ Key Note on Solving Ratio & Proportion Problems  
📌 Solving ratio-based problems (especially **salary-related ones**) involves forming equations step by step based on what we need.

---

✅📚💡
